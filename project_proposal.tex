\documentclass[twoside,11pt]{article}

% Any additional packages needed should be included after jmlr2e.
% Note that jmlr2e.sty includes epsfig, amssymb, natbib and graphicx,
% and defines many common macros, such as 'proof' and 'example'.
%
% It also sets the bibliographystyle to plainnat; for more information on
% natbib citation styles, see the natbib documentation, a copy of which
% is archived at http://www.jmlr.org/format/natbib.pdf

\usepackage{jmlr2e}
%\usepackage{parskip}

% Definitions of handy macros can go here
\newcommand{\dataset}{{\cal D}}
\newcommand{\fracpartial}[2]{\frac{\partial #1}{\partial  #2}}
% Heading arguments are {volume}{year}{pages}{submitted}{published}{author-full-names}

% Short headings should be running head and authors last names
\ShortHeadings{95-845: AAMLP Proposal}{Redd and Roberts}
\firstpageno{1}

\begin{document}

\title{Heinz 95-845: Project Proposal}

\author{\name Andrew Redd \email aredd/aredd@andrew.cmu.edu \\
       \addr Heinz College\\
       Carnegie Mellon University\\
       Pittsburgh, PA, United States \\
       \AND
       \name Alexander Roberts \email adr1/adr1@andrew.cmu.edu \\
       \addr Heinz College\\
       Carnegie Mellon University\\
       Pittsburgh, PA, United States}
\maketitle

\section{Proposal Details (10 points)} \label{details}

\subsection{What is your proposed analysis? What are the likely outcomes?}

We would like to predict whether a Chicago SRW future is overvalued or undervalued on a given day based on weather data sources. We propose to collect and clean weather data from several major wheat growing areas and use a RNN with an LSTM layer to predict daily over/undervaluation. 

Because the future price today is already likely a strong predictor of final future value, we predict the difference on a given day from the final future value. If our selection of weather covariates and modeling techniques control for additional factors not accounted for by other participants in the futures market, the model would expose unrealized gains.  

\subsection{Why is your proposed analysis important?}

Speculation on commodity futures is extremely risky but does come with a strong upside if it can be moderately improved above random. Such a model if successful would provide an arbitrage opportunity to an investor. 

\subsection{How will your analysis contribute to existing work? Provide references.}

While there are references to using neural nets to predict futures prices, we have not come across any published papers that incorporate weather data in their models.  There is a good chance this is due them being proprietary as individuals would not necessarily want to publish.  


\subsection{Describe the data. Please also define Y outcome(s), U treatment, V covariates, W population as applicable.}

\subsubsection{Outcome Variable (Y)}
Final Future Price - Closing future price

\subsubsection{Covariates Variables (V)}

- Final Future Closing Date - Closing future date\\  
- Closing future price\\
- Daily volume of future

\subsubsection{Example data points include}
\begin{itemize}
  \item Average temperature in wheat growing region 
  \item Distance from historic temperatures (1.2 SDs higher than historic temperatures)
  \item Average precipitation in wheat growing region
  \item Distance from historic precipitation
  \item Average humidity in wheat growing region
  \item Distance from historic humidity
  \item Average wind speed in wheat growing region
  \item Distance from historic wind speed
  \item Average barometric pressure in wheat growing region
  \item Mode weather condition in the wheat growing region
\end{itemize}


\subsection{What evaluation measures are appropriate for the analysis? Which measures will you use?}

The evaluation measure will be a train/test split on the data that will test the investment strategy uncovered by the model. The primary indicator of model performance will be expected profit under testing circumstances.

\subsection{What study design, pre-processing, and machine learning methods do you intend to use? Justify that the analysis is of appropriate size for a course project.}

The analysis is designed to account for the timing of events and prices to predict the difference in price. We will investigate RNNs and other time-series prediction methods (ARIMA, etc.). The complexity of structuring and cleaning the weather data and futures data will be the primary challenge.

\subsection{What are possible limitations of the study?}

Possible limitations that we foresee in the study is in the 'stitching' together of the futures contract data.  To analyze longer-term futures prices, one needs to concatenate various individual futures contracts by selecting the appropriate day (roll-day) and any adjustments to price to eliminate error-inducing jumps in the continuous contract price history.  The free dataset that we are using provides one roll-day/price-adjustment rule, while the professional dataset (paid subscription needed) offers 14 different rules.  Ideally, one would incorporate a rule that matches their actual roll trading strategy.  So, we are not sure how having one roll-date/price-adjustment rule available to us will affect the continuous contract data we are using and thus our model.\\
\indent Another limitation would be how much of the weather information we are adding is already incorporated in the futures price, especially as we are using daily closing prices.




\bibliography{}
%\appendix
%\section*{Appendix A.}
%Some more details about those methods, so we can actually reproduce them.

\end{document}
